% Vorlage für das Dokument laden:
%   - scrartl  -> Artikel
%   - scrreprt -> Bericht
%   - scrbook  -> Buch

\documentclass[a4paper, 11pt]{scrartcl}

% Hier befinden sich alle Metadaten; Pakete, Einstellungen und die sonstigen Ressourcen
% Pakete laden

%%%%%%%%%%%%%%%%%%%%%%%%%%%%%%%%%%%%%%%%%%%%%%%%%%%

% TODO: Der Fehler hier muss noch behoben werden (das hier funktioniert nicht)

% Einbinden eines Pakets für den Zitationsstil
%\usepackage{natbib}

% Pakete laden (die restlichen)

%%%%%%%%%%%%%%%%%%%%%%%%%%%%%%%%%%%%%%%%%%%%%%%%%%%

% Bildoptionen?
\usepackage{pict2e}

% Latex - Symbole
\usepackage{latexsym}

% mathematische Symbole
% inkompatibel wegen biblatex
%\usepackage[mathletters]{ucs}

% Zusätzliche Kodierungen
\usepackage{inconsolata}

% Nicht - umbrechbare Leerzeichen
\usepackage{newunicodechar}

% Abbildungen
\usepackage{tikz}

% Weitere TikZ - Erweiterungen
\usetikzlibrary{chains,shapes.multipart}

% Erweiterungen für die Abbildungen mit Tikz
\usetikzlibrary{shapes,arrows,positioning}

% Zitationen
% inkompatibel wegen biblatex
%\usepackage{cite}

% Abbildungen nebeneinander stellen und Text anordnen
\usepackage{subcaption}

% Mathematische Symbole
\usepackage{dsfont}

% Mathematische Symbole
\usepackage{amssymb}

% Schöner Aufzählungspunkt; und andere schöne Textelemente
\usepackage{textcomp}

% Pfeiloptimierung; und ähnliches
\usepackage{stmaryrd}

% Blindtext (zur Hilfe beim Schreiben)
\usepackage{blindtext}

% Text-Boxen (bei den algorithmischen Problemen)
\usepackage{tcolorbox}

% Grafiken
\usepackage{graphicx}

% Eingabecodierung utf8
\usepackage[utf8]{inputenc}

% Ausgabecodierung von Sonderzeichung
\usepackage[T1]{fontenc}

% Seitenränder anpassen
\usepackage[left=1.5cm, right=1.5cm]{geometry}

% Silbentrennung
\usepackage[ngerman]{babel}

% Farbiger Text
\usepackage{xcolor}
\usepackage{colortbl}

% Optimierung von Grafiken
\usepackage{changepage}
\usepackage{adjustbox}

% Erweiterungen für den Mathemodus
\usepackage{amsmath}

% Erweiterung: Umgebungen für Definitionen etc
\usepackage{amsthm}

% Erweiterung: Mehr Symbole
\usepackage{amssymb}

% Erweiterung: Mathefonts
\usepackage{amsfonts}

% Noch mehr Mathe:
\usepackage[centerdots]{mathtools}

% Für Code/Pseudocode
\usepackage{listings}

% Literatur
%\usepackage{bibtex}

% ¿Eventuell erforderlich?
\usepackage{array}

%Farbauswahl
\usepackage{xcolor}

% Klickbare Verweise
\usepackage{hyperref}

% Schönere Schriftart
\usepackage{lmodern}

% Tabellen ausrichten
\usepackage{float}

% Bemerkung
\usepackage{amsthm} 

% Mehr (mathematische) Symbole
\usepackage{tipa}

% Um schönere Brüche darstellen zu können
\usepackage{nicefrac}

% um Hyperlinks setzen zu können
\usepackage{hyperref}

% um Todo Anmerkungen einfügen zu können
\usepackage{todonotes}
\usepackage{enumitem}
\usepackage{booktabs}

% damit wir Algorithmen in Pseudocode einfach setzen können
\usepackage{algorithm}
\usepackage{algpseudocode}

%%%%%%%%%%%%%%%%%%%%%%%%%%%%%%%%%%%%%%%%%%%%%%%%%%%

%%%%%%%%%%%%%%%%%%%%%%%%%%%%%%%%%%%%%%%%%%%%%%%%%%%

% Einstellungen für das Dokument

%%%%%%%%%%%%%%%%%%%%%%%%%%%%%%%%%%%%%%%%%%%%%%%%%%%

% Eine neue Farbe für keywords im Code und für links
\definecolor{codeblue}{HTML}{4178f2}

%Farbe für Textblocktitel
\definecolor{lieblingsfarbe}{cmyk}{79,0.11,1,0.25}

% Entfernt den roten Rahmen um klickbare Links und macht Links blau
\hypersetup{hidelinks, colorlinks=true, allcolors=codeblue}

% Stildefinition für Aufzählungspunkte
\setlist[itemize]{label=$\bullet$, leftmargin=*, nosep}


% Zitatstil: Sinnvolle Alternativen: alpha, abbr [bibtex nach dem Ändern ausführen]
% Nicht mehr notwendig; anders gelöst
%\bibliographystyle{plain}

% Vordefinierte Umgebungen für Theoreme/Lemmas/Definitionen
\theoremstyle{plain} % Alle newtheorems die hier nach kommen bekommen den 'plain' Stil
\newtheorem{theorem}{Theorem}
\newtheorem{satz}[theorem]{Satz}
\newtheorem{fundamentalsatz}[theorem]{Fundamentalsatz}
\newtheorem{intuition}[theorem]{Intuition}
\newtheorem{intuitionUndBeispiel}[theorem]{Intuition und Beispiel}
\newtheorem{intuitionUndDefinition}[theorem]{Intuition und Definition}
\newtheorem{intuitionDefinitionUndSatz}[theorem]{Intuition, Definition und Satz}
\newtheorem{beispiel}[theorem]{Beispiel}
\newtheorem{lemma}[theorem]{Lemma}
\newtheorem{korollar}[theorem]{Korollar}
\newtheorem{definition}[theorem]{Definition}
\newtheorem{bemerkung}[theorem]{Bemerkung}
\newtheorem{vorbemerkung}[theorem]{Vorbemerkung}
\newtheorem{amortisierteAnalyse}[theorem]{Amortisierte Analyse}

% Beweisumgebung im gleichen Stil wie die anderen Umgebungen
\newenvironment{beweis}[1][\textbf{Beweis}]{%
	\begin{proof}[#1]
	}{%
	\end{proof}
}

%%%%%%%%%%%%%%%%%%%%%
%(¡Diese beiden Funktionen sind experimentell!)

% Befehlsdefinition für dynamische Größenanpassung des Bildes
\newcommand{\dynamicscale}[1]{%
	\includegraphics[width=\minof{\textwidth}{\widthof{\includegraphics{#1}}-\parindent}]{#1}%
}

% Befehlsdefinition für dynamische Größenanpassung der TiKZ-Grafik
\newcommand{\dynamictikz}[1]{%
	\begin{tikzpicture}[scale=1]
		\draw (0,0) rectangle (\minof{\textwidth}{\widthof{#1}}-\parindent, \heightof{#1});
		#1
	\end{tikzpicture}%
}

% Stil für TikZ - Grafiken
\tikzset{
	heapnode/.style={draw, circle, inner sep=0pt,
		minimum size=6mm},
	heaparrow/.style={-latex, thick},
}

%%%%%%%%%%%%%%%%%%%%%

% Stelle Theorem/Lemma/Definitionsnummerierung die Abschnittsnummer voran
\numberwithin{theorem}{section}

% Ersetze U+202F durch ein normales Leerzeichen
\newunicodechar{ }{ }

% Ersetze U+0308 durch das LaTeX-Äquivalent für das Diaeresis (z.B., \"{o} für ö)
\newunicodechar{̈}{\"}

%%%%%%%%%%%%%%%%%%%%%%%%%%%%%%%%%%%%%%%%%%%%%%%%%%%

% Definiere Sprache 'pseudocode' für Codelistings
\lstdefinelanguage{pseudocode}{
	keywords={
		struct,
		null,
		function,
		error,
		nil,
		if,
		then,
		else,
		for,
		while,
		do,
		in,
		True,
		False,
		Array,
		set,
		repeat,
		break,
		continue,
		Eingabe,
		Ausgabe,
		return,
		print,
		exit
	},
	sensitive=false,
	delim=[l][keywordstyle]{:},
	comment=[l][commentstyle]{\#}
}

% Generelle Einstellungen
\lstset{%
	frame=single,
	framesep=0mm,
	framexleftmargin=7mm,
	xleftmargin=8mm,
	framerule=1.5pt,
	rulecolor=\color{black},
	backgroundcolor=\color{white},
	basicstyle=\ttfamily,
	keywordstyle=\bfseries\color{codeblue},
	commentstyle=\color{gray},
	numbers=left,
	stepnumber=1,
	numbersep=1mm,
	numberstyle=\sffamily\color{gray!80!black}\footnotesize,
	numberblanklines=true,
	escapeinside={\%*}{*)},
	inputencoding=utf8
}

% Eine weitere Stufe für Unterkapitel

% Definiere das Gliederungselement Subsubsubsection
\makeatletter
\newcommand\subsubsubsection{\@startsection{subsubsubsection}{4}{\z@}%
	{-3.25ex \@plus -1ex \@minus -.2ex}%
	{1.5ex \@plus .2ex}%
	{\normalfont\normalsize\bfseries}}
\newcounter{subsubsubsection}[subsubsection]
\renewcommand\thesubsubsubsection{\thesubsubsection.\@arabic\c@subsubsubsection}
\newcommand*\l@subsubsubsection{\@dottedtocline{3}{10.0em}{4.1em}}
\newcommand*{\subsubsubsectionmark}[1]{}
\makeatother

% Füge das Gliederungselement dem Inhaltsverzeichnis hinzu
\setcounter{secnumdepth}{4}
\setcounter{tocdepth}{4}

% Dieser von Stackoverflow stammende Block sorgt dafür, dass Umlaute in Code-Listings funktionieren.
% Quelle:  https://tex.stackexchange.com/questions/24528/having-problems-with-listings-and-utf-8-can-it-be-fixed
\lstset{
	inputencoding = utf8,  % Input encoding
	extendedchars = true,  % Extended ASCII
	literate      =        % Support additional characters
	{á}{{\'a}}1  {é}{{\'e}}1  {í}{{\'i}}1 {ó}{{\'o}}1  {ú}{{\'u}}1
	{Á}{{\'A}}1  {É}{{\'E}}1  {Í}{{\'I}}1 {Ó}{{\'O}}1  {Ú}{{\'U}}1
	{à}{{\`a}}1  {è}{{\`e}}1  {ì}{{\`i}}1 {ò}{{\`o}}1  {ù}{{\`u}}1
	{À}{{\`A}}1  {È}{{\`E}}1  {Ì}{{\`I}}1 {Ò}{{\`O}}1  {Ù}{{\`U}}1
	{ä}{{\"a}}1  {ë}{{\"e}}1  {ï}{{\"i}}1 {ö}{{\"o}}1  {ü}{{\"u}}1
	{Ä}{{\"A}}1  {Ë}{{\"E}}1  {Ï}{{\"I}}1 {Ö}{{\"O}}1  {Ü}{{\"U}}1
	{â}{{\^a}}1  {ê}{{\^e}}1  {î}{{\^i}}1 {ô}{{\^o}}1  {û}{{\^u}}1
	{Â}{{\^A}}1  {Ê}{{\^E}}1  {Î}{{\^I}}1 {Ô}{{\^O}}1  {Û}{{\^U}}1
	{œ}{{\oe}}1  {Œ}{{\OE}}1  {æ}{{\ae}}1 {Æ}{{\AE}}1  {ß}{{\ss}}1
	{ẞ}{{\SS}}1  {ç}{{\c{c}}}1 {Ç}{{\c{C}}}1 {ø}{{\o}}1  {Ø}{{\O}}1
	{å}{{\aa}}1  {Å}{{\AA}}1  {ã}{{\~a}}1  {õ}{{\~o}}1 {Ã}{{\~A}}1
	{Õ}{{\~O}}1  {ñ}{{\~n}}1  {Ñ}{{\~N}}1  {¿}{{?`}}1  {¡}{{!`}}1
	{°}{{\textdegree}}1 {º}{{\textordmasculine}}1 {ª}{{\textordfeminine}}1
	{£}{{\pounds}}1  {©}{{\copyright}}1  {®}{{\textregistered}}1
	{«}{{\guillemotleft}}1  {»}{{\guillemotright}}1  {Ð}{{\DH}}1  {ð}{{\dh}}1
	{Ý}{{\'Y}}1    {ý}{{\'y}}1    {Þ}{{\TH}}1    {þ}{{\th}}1    {Ă}{{\u{A}}}1
	{ă}{{\u{a}}}1  {Ą}{{\k{A}}}1  {ą}{{\k{a}}}1  {Ć}{{\'C}}1    {ć}{{\'c}}1
	{Č}{{\v{C}}}1  {č}{{\v{c}}}1  {Ď}{{\v{D}}}1  {ď}{{\v{d}}}1  {Đ}{{\DJ}}1
	{đ}{{\dj}}1    {Ė}{{\.{E}}}1  {ė}{{\.{e}}}1  {Ę}{{\k{E}}}1  {ę}{{\k{e}}}1
	{Ě}{{\v{E}}}1  {ě}{{\v{e}}}1  {Ğ}{{\u{G}}}1  {ğ}{{\u{g}}}1  {Ĩ}{{\~I}}1
	{ĩ}{{\~\i}}1   {Į}{{\k{I}}}1  {į}{{\k{i}}}1  {İ}{{\.{I}}}1  {ı}{{\i}}1
	{Ĺ}{{\'L}}1    {ĺ}{{\'l}}1    {Ľ}{{\v{L}}}1  {ľ}{{\v{l}}}1  {Ł}{{\L{}}}1
	{ł}{{\l{}}}1   {Ń}{{\'N}}1    {ń}{{\'n}}1    {Ň}{{\v{N}}}1  {ň}{{\v{n}}}1
	{Ő}{{\H{O}}}1  {ő}{{\H{o}}}1  {Ŕ}{{\'{R}}}1  {ŕ}{{\'{r}}}1  {Ř}{{\v{R}}}1
	{ř}{{\v{r}}}1  {Ś}{{\'S}}1    {ś}{{\'s}}1    {Ş}{{\c{S}}}1  {ş}{{\c{s}}}1
	{Š}{{\v{S}}}1  {š}{{\v{s}}}1  {Ť}{{\v{T}}}1  {ť}{{\v{t}}}1  {Ũ}{{\~U}}1
	{ũ}{{\~u}}1    {Ū}{{\={U}}}1  {ū}{{\={u}}}1  {Ů}{{\r{U}}}1  {ů}{{\r{u}}}1
	{Ű}{{\H{U}}}1  {ű}{{\H{u}}}1  {Ų}{{\k{U}}}1  {ų}{{\k{u}}}1  {Ź}{{\'Z}}1
	{ź}{{\'z}}1    {Ż}{{\.Z}}1    {ż}{{\.z}}1    {Ž}{{\v{Z}}}1
	% ¿ and ¡ are not correctly displayed if inconsolata font is used
	% together with the lstlisting environment. Consider typing code in
	% external files and using \lstinputlisting to display them instead.      
}

% Definiere einige Stile für TikZ - Knoten
\tikzset{
	long node/.style={draw, rounded corners, minimum width=7.5cm, minimum height=1cm},
	short node/.style={draw, rounded corners, minimum width=2.5cm, minimum height=1cm},
	square node/.style={draw, rounded corners, minimum width=2.5cm, minimum height=2.5cm},
	label node/.style={font=\small, align=center}
}

% Abkürzungen ("Makros") können mit \newcommand definiert werden
% Symbole für die gängigen Mengen
\newcommand{\N}{\mathbb{N}}
\newcommand{\Z}{\mathbb{Z}}
\newcommand{\Zp}{\Z_{\geq 0}}
\newcommand{\Q}{\mathbb{Q}}
\newcommand{\Qp}{\Q_{\geq 0}}
\newcommand{\R}{\mathbb{R}}
\newcommand{\Rp}{\R_{\geq 0}}
\newcommand{\Rmn}{\R^{m\times n}}

% Hübscheres Epsilon
\newcommand{\eps}{\varepsilon}

% Einführung von TikZ - Bibliotheken
\usetikzlibrary{positioning,fit,shapes.geometric}

%%%%%%%%%%%%%%%%%%%%%%%%%%%%%%%%%%%%%%%%%%%%%%%%%%%
% Metadaten: Autor, Titel, etc.
\author{Yavuzâlp Dal}
\title{Einf\"uhrung in die Kerndichtesch\"atzung}
\subject{Schriftliche Ausarbeitung zum Seminarvortrag}
\def \vorlesung {Seminar: Angewandte Analysis}
\def \semester {Wintersemester 2025/26}
\def \dozent {Prof. Dr. Holger Schwender}
\date{Stand: \today}
%%%%%%%%%%%%%%%%%%%%%%%%%%%%%%%%%%%%%%%%%%%%%%%%%%%

\usepackage[utf8]{inputenc}

\usepackage[ngerman]{babel}
\usepackage[backend=bibtex, style=alphabetic]{biblatex}
\addbibresource{Ausarbeitung.bib}

% Definiere eine Bedingung, die steuert, ob der Text angezeigt wird oder nicht
\newif\ifshowtext
% Setze die Bedingung auf "wahr" (Text wird standardmäßig angezeigt)
\showtexttrue

\begin{document}
	
% TikZ - Bibliothek
\usetikzlibrary{arrows.meta}
\usetikzlibrary{shapes,arrows.meta,positioning}

\maketitle

\section{Einleitung \& Motivation}

\ifshowtext
% Inhaltsverzeichnis erzeugen
\tableofcontents
\fi

		
\subsection*{Warum nichtparametrisch?}
	Klassische parametrische Verfahren (z.\,B. unter Annahme einer Normalverteilung) sind oft zu starr. Wenn die wahre Verteilung unbekannt ist, liefert die nichtparametrische Schätzung der Dichtefunktion $f$ oft bessere Einblicke in die Datenstruktur (z.\,B. Schiefe, Multimodalität) als die reine Verteilungsfunktion $F$.
	\\
	Ferner k\"onnen wir eine Verteilung normalerweise nicht beweisen. Wir nehmen sie lediglich an. Folglich erzeugen wir hier eine extreme Kopplung zwischen (potentiell fehlerhaften) Interpretationen und unserer Sch\"atzung. Offensichtlich ist es folglich f\"ur eine produktive Arbeit sinnvoll, diese Kopplung m\"oglichst abzuschw\"achen; wobei wir im Nachfolgenden gesehen haben werden, dass sich bestimmte (schwache) Annahmen wie die Stetigkeit nicht zwingend vermeiden lassen, jedoch deren Einfluss i.d.R. nicht zu gro\ss\ ist.
		
\subsection*{Zielsetzung}
Wir suchen eine Annäherung an den Wert $f(x)$ an einer Stelle $x$, ohne eine bestimmte parametrische Familie vorauszusetzen.
	
\section{Der Rosenblatt-Schätzer (Die theoretische Basis)}
Der Ausgangspunkt für moderne Dichteschätzer ist der Ansatz von Rosenblatt (1956).
	
\begin{itemize}
	\item \textbf{Idee:} Da $f(x) = F'(x)$, nutzt man den Differenzenquotienten der empirischen Verteilungsfunktion $F_n$.
	\item \textbf{Der Schätzer:}
	      \begin{equation}
	      	f_n(x) = \frac{F_n(x+h) - F_n(x-h)}{2h}
	      \end{equation}
	      Dies entspricht der relativen Häufigkeit von Datenpunkten im Intervall $(x-h, x+h]$, geteilt durch die Länge $2h$.
	\item \textbf{Konsistenz:} Damit der Schätzer gegen die wahre Dichte konvergiert (MSE $\to 0$), muss für den Stichprobenumfang $n \to \infty$ gelten:
	      \begin{enumerate}
	      	\item Die Bandbreite $h \to 0$ (um den Bias zu reduzieren).
	      	\item Das Produkt $nh \to \infty$ (um die Varianz zu reduzieren).
	      \end{enumerate}
\end{itemize}
	
\section{Das Histogramm (Der Klassiker)}
Das Histogramm ist die älteste Methode, hat aber methodische Schwächen für präzise Analysen.
	
\begin{itemize}
	\item \textbf{Funktionsweise:} Zerlegung des Datenraums in Boxen (Bins) der Breite $h$. Die Höhe der Box entspricht der relativen Häufigkeit $\frac{n_k}{n}$ normiert durch $h$.
	\item \textbf{Probleme:}
	      \begin{enumerate}
	      	\item \textbf{Unstetigkeit:} Die geschätzte Dichte ist eine Treppenfunktion, obwohl die wahre Natur oft glatt (stetig differenzierbar) ist.
	      	\item \textbf{Subjektivität:} Form und Aussage hängen stark vom Startpunkt $x_0$ und der Klassenbreite $h$ ab.
	      \end{enumerate}
	\item \textbf{Optimale Bandbreite ($h$):} Die Wahl von $h$ ist ein Balanceakt (Bias-Varianz-Tradeoff). Ein zu kleines $h$ erzeugt eine ,,zackige`` Kurve (hohe Varianz), ein zu großes $h$ glättet wichtige Details weg (hoher Bias).
	      \begin{itemize}
	      	\item \textit{Faustregel (Scott):} $h \approx 3.49 s n^{-1/3}$ (basiert auf Normalverteilungsannahme).
	      	\item \textit{Robuste Regel (Freedman-Diaconis):} Nutzt den Interquartilsabstand, um Ausreißer weniger zu gewichten:
	      	      \[ h^* = 2(x_{0.75} - x_{0.25})n^{-1/3} \]
	      \end{itemize}
\end{itemize}
	
\section{Kernschätzer (Kernel Density Estimation - KDE)}
Um die Glattheit zu garantieren, verallgemeinert man den Rosenblatt-Ansatz durch sogenannte Kerne.
		
\begin{itemize}
	\item \textbf{Der Schätzer:}
	      \begin{equation}
	      	\hat{f}_n(x) = \frac{1}{nh} \sum_{i=1}^{n} K\left(\frac{x - X_i}{h}\right)
	      \end{equation}
	      Jeder Datenpunkt $X_i$ bekommt eine kleine ,,Glockenkurve`` (Kern $K$), und die Summe dieser Kurven ergibt die Gesamtdichte.
	      		    
	\item \textbf{Wahl des Kerns ($K$):} Der Kern muss zu 1 integrieren ($\int K(x)dx = 1$). Gängige Kerne sind:
	      \begin{itemize}
	      	\item \textit{Rechteck-Kern:} Entspricht dem naiven Rosenblatt-Schätzer.
	      	\item \textit{Gauß-Kern:} Standardnormalverteilung (glatt und differenzierbar).
	      	\item \textit{Epanechnikov-Kern:} Parabolisch; dieser Kern ist theoretisch optimal, da er den integrierten mittleren quadratischen Fehler (IMSE) minimiert.
	      \end{itemize}
	      		    
	\item \textbf{Bandbreitenwahl (Silverman's Rule):} Die Bandbreite $h$ ist viel entscheidender als die Kernform. Eine Standard-Faustformel (für Gauß-Kerne) ist:
	      \begin{equation}
	      	h_{opt} \approx 1.06 \sigma n^{-1/5}
	      \end{equation}
	      Oft wird $\sigma$ robust durch den Quartilsabstand geschätzt ($\hat{h}_{opt} \approx 0.79 Q n^{-1/5}$), um Oversmoothing zu vermeiden.
\end{itemize}
		
\section{Nichtparametrische Regression}
Hier verlassen wir die Dichteschätzung und betrachten den Zusammenhang zwischen zwei Variablen $X$ und $Y$: $Y = m(X) + \epsilon$.
	
\subsection{Watson-Nadaraya Schätzer (Allgemeine Regression)}
Wenn wir keine Formel für $m(x)$ kennen, schätzen wir den Wert als gewichteten Mittelwert der umliegenden $Y$-Werte:
\begin{equation}
	\hat{m}_{WN}(x) = \frac{\sum_{i=1}^n K\left(\frac{x - X_i}{h}\right) Y_i}{\sum_{i=1}^n K\left(\frac{x - X_i}{h}\right)}
\end{equation}
Datenpunkte $X_i$, die nahe an $x$ liegen, erhalten durch den Kern $K$ ein hohes Gewicht.
	
\subsection{Robuste Lineare Regression (Theil-Schätzer)}
Selbst wenn wir einen linearen Trend ($Y = \alpha + \beta X$) vermuten, ist die klassische Methode der kleinsten Quadrate (OLS) anfällig für Ausreißer. Das Buch stellt robuste Alternativen vor:
	
\begin{enumerate}
	\item \textbf{Theil-Methode I:} Man teilt die Daten in zwei Hälften und berechnet Steigungen zwischen Paaren $(i, i+n/2)$. Der Schätzer ist der Median dieser Steigungen.
	\item \textbf{Theil-Sen-Schätzer (Methode II):} Dies ist die präzisere Variante. Man berechnet die Steigung zwischen \textbf{allen} möglichen Paaren $i < j$:
	      \begin{equation}
	      	H_{ij} = \frac{Y_j - Y_i}{X_j - X_i}
	      \end{equation}
	      Der Schätzer für den Anstieg $\beta$ ist der \textbf{Median} all dieser Steigungen ($H_{ij}$). Das macht ihn extrem robust.
	\item \textbf{Hypothesentests:} Zum Testen von $\beta$ wird auf rangbasierte Verfahren wie Kendalls $\tau$ (Tau) zurückgegriffen, da diese verteilungsfrei sind.
\end{enumerate}
	
\section{Zusammenfassung für das Seminar}
\begin{itemize}
	\item \textbf{Flexibilität:} Nichtparametrische Methoden (KDE, Kernel-Regression) passen sich den Daten an (,,let the data speak``) und zwingen ihnen keine Form auf.
	\item \textbf{Parameter $h$:} Die Wahl der Bandbreite ist der kritischste Schritt. Es ist ein Kompromiss zwischen Rauschen (zu kleines $h$) und Informationsverlust (zu großes $h$).
	\item \textbf{Robustheit:} Verfahren wie der Theil-Sen-Schätzer bieten mächtige Alternativen zur klassischen Regression, besonders wenn Daten Ausreißer enthalten oder nicht normalverteilt sind.
\end{itemize}
			
\ifshowtext
% Referenzen
\printbibliography
\fi
	
\end{document}
