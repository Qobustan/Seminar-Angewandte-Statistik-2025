% Pakete laden

%%%%%%%%%%%%%%%%%%%%%%%%%%%%%%%%%%%%%%%%%%%%%%%%%%%

% TODO: Der Fehler hier muss noch behoben werden (das hier funktioniert nicht)

% Einbinden eines Pakets für den Zitationsstil
%\usepackage{natbib}

% Pakete laden (die restlichen)

%%%%%%%%%%%%%%%%%%%%%%%%%%%%%%%%%%%%%%%%%%%%%%%%%%%

% Bildoptionen?
\usepackage{pict2e}

% Latex - Symbole
\usepackage{latexsym}

% mathematische Symbole
% inkompatibel wegen biblatex
%\usepackage[mathletters]{ucs}

% Zusätzliche Kodierungen
\usepackage{inconsolata}

% Nicht - umbrechbare Leerzeichen
\usepackage{newunicodechar}

% Abbildungen
\usepackage{tikz}

% Weitere TikZ - Erweiterungen
\usetikzlibrary{chains,shapes.multipart}

% Erweiterungen für die Abbildungen mit Tikz
\usetikzlibrary{shapes,arrows,positioning}

% Zitationen
% inkompatibel wegen biblatex
%\usepackage{cite}

% Abbildungen nebeneinander stellen und Text anordnen
\usepackage{subcaption}

% Mathematische Symbole
\usepackage{dsfont}

% Mathematische Symbole
\usepackage{amssymb}

% Schöner Aufzählungspunkt; und andere schöne Textelemente
\usepackage{textcomp}

% Pfeiloptimierung; und ähnliches
\usepackage{stmaryrd}

% Blindtext (zur Hilfe beim Schreiben)
\usepackage{blindtext}

% Text-Boxen (bei den algorithmischen Problemen)
\usepackage{tcolorbox}

% Grafiken
\usepackage{graphicx}

% Eingabecodierung utf8
\usepackage[utf8]{inputenc}

% Ausgabecodierung von Sonderzeichung
\usepackage[T1]{fontenc}

% Seitenränder anpassen
\usepackage[left=1.5cm, right=1.5cm]{geometry}

% Silbentrennung
\usepackage[ngerman]{babel}

% Farbiger Text
\usepackage{xcolor}
\usepackage{colortbl}

% Optimierung von Grafiken
\usepackage{changepage}
\usepackage{adjustbox}

% Erweiterungen für den Mathemodus
\usepackage{amsmath}

% Erweiterung: Umgebungen für Definitionen etc
\usepackage{amsthm}

% Erweiterung: Mehr Symbole
\usepackage{amssymb}

% Erweiterung: Mathefonts
\usepackage{amsfonts}

% Noch mehr Mathe:
\usepackage[centerdots]{mathtools}

% Für Code/Pseudocode
\usepackage{listings}

% Literatur
%\usepackage{bibtex}

% ¿Eventuell erforderlich?
\usepackage{array}

%Farbauswahl
\usepackage{xcolor}

% Klickbare Verweise
\usepackage{hyperref}

% Schönere Schriftart
\usepackage{lmodern}

% Tabellen ausrichten
\usepackage{float}

% Bemerkung
\usepackage{amsthm} 

% Mehr (mathematische) Symbole
\usepackage{tipa}

% Um schönere Brüche darstellen zu können
\usepackage{nicefrac}

% um Hyperlinks setzen zu können
\usepackage{hyperref}

% um Todo Anmerkungen einfügen zu können
\usepackage{todonotes}
\usepackage{enumitem}
\usepackage{booktabs}

% damit wir Algorithmen in Pseudocode einfach setzen können
\usepackage{algorithm}
\usepackage{algpseudocode}

%%%%%%%%%%%%%%%%%%%%%%%%%%%%%%%%%%%%%%%%%%%%%%%%%%%

%%%%%%%%%%%%%%%%%%%%%%%%%%%%%%%%%%%%%%%%%%%%%%%%%%%

% Einstellungen für das Dokument

%%%%%%%%%%%%%%%%%%%%%%%%%%%%%%%%%%%%%%%%%%%%%%%%%%%

% Eine neue Farbe für keywords im Code und für links
\definecolor{codeblue}{HTML}{4178f2}

%Farbe für Textblocktitel
\definecolor{lieblingsfarbe}{cmyk}{79,0.11,1,0.25}

% Entfernt den roten Rahmen um klickbare Links und macht Links blau
\hypersetup{hidelinks, colorlinks=true, allcolors=codeblue}

% Stildefinition für Aufzählungspunkte
\setlist[itemize]{label=$\bullet$, leftmargin=*, nosep}


% Zitatstil: Sinnvolle Alternativen: alpha, abbr [bibtex nach dem Ändern ausführen]
% Nicht mehr notwendig; anders gelöst
%\bibliographystyle{plain}

% Vordefinierte Umgebungen für Theoreme/Lemmas/Definitionen
\theoremstyle{plain} % Alle newtheorems die hier nach kommen bekommen den 'plain' Stil
\newtheorem{theorem}{Theorem}
\newtheorem{satz}[theorem]{Satz}
\newtheorem{fundamentalsatz}[theorem]{Fundamentalsatz}
\newtheorem{intuition}[theorem]{Intuition}
\newtheorem{intuitionUndBeispiel}[theorem]{Intuition und Beispiel}
\newtheorem{intuitionUndDefinition}[theorem]{Intuition und Definition}
\newtheorem{intuitionDefinitionUndSatz}[theorem]{Intuition, Definition und Satz}
\newtheorem{beispiel}[theorem]{Beispiel}
\newtheorem{lemma}[theorem]{Lemma}
\newtheorem{korollar}[theorem]{Korollar}
\newtheorem{definition}[theorem]{Definition}
\newtheorem{bemerkung}[theorem]{Bemerkung}
\newtheorem{vorbemerkung}[theorem]{Vorbemerkung}
\newtheorem{amortisierteAnalyse}[theorem]{Amortisierte Analyse}

% Beweisumgebung im gleichen Stil wie die anderen Umgebungen
\newenvironment{beweis}[1][\textbf{Beweis}]{%
	\begin{proof}[#1]
	}{%
	\end{proof}
}

%%%%%%%%%%%%%%%%%%%%%
%(¡Diese beiden Funktionen sind experimentell!)

% Befehlsdefinition für dynamische Größenanpassung des Bildes
\newcommand{\dynamicscale}[1]{%
	\includegraphics[width=\minof{\textwidth}{\widthof{\includegraphics{#1}}-\parindent}]{#1}%
}

% Befehlsdefinition für dynamische Größenanpassung der TiKZ-Grafik
\newcommand{\dynamictikz}[1]{%
	\begin{tikzpicture}[scale=1]
		\draw (0,0) rectangle (\minof{\textwidth}{\widthof{#1}}-\parindent, \heightof{#1});
		#1
	\end{tikzpicture}%
}

% Stil für TikZ - Grafiken
\tikzset{
	heapnode/.style={draw, circle, inner sep=0pt,
		minimum size=6mm},
	heaparrow/.style={-latex, thick},
}

%%%%%%%%%%%%%%%%%%%%%

% Stelle Theorem/Lemma/Definitionsnummerierung die Abschnittsnummer voran
\numberwithin{theorem}{section}

% Ersetze U+202F durch ein normales Leerzeichen
\newunicodechar{ }{ }

% Ersetze U+0308 durch das LaTeX-Äquivalent für das Diaeresis (z.B., \"{o} für ö)
\newunicodechar{̈}{\"}

%%%%%%%%%%%%%%%%%%%%%%%%%%%%%%%%%%%%%%%%%%%%%%%%%%%

% Definiere Sprache 'pseudocode' für Codelistings
\lstdefinelanguage{pseudocode}{
	keywords={
		struct,
		null,
		function,
		error,
		nil,
		if,
		then,
		else,
		for,
		while,
		do,
		in,
		True,
		False,
		Array,
		set,
		repeat,
		break,
		continue,
		Eingabe,
		Ausgabe,
		return,
		print,
		exit
	},
	sensitive=false,
	delim=[l][keywordstyle]{:},
	comment=[l][commentstyle]{\#}
}

% Generelle Einstellungen
\lstset{%
	frame=single,
	framesep=0mm,
	framexleftmargin=7mm,
	xleftmargin=8mm,
	framerule=1.5pt,
	rulecolor=\color{black},
	backgroundcolor=\color{white},
	basicstyle=\ttfamily,
	keywordstyle=\bfseries\color{codeblue},
	commentstyle=\color{gray},
	numbers=left,
	stepnumber=1,
	numbersep=1mm,
	numberstyle=\sffamily\color{gray!80!black}\footnotesize,
	numberblanklines=true,
	escapeinside={\%*}{*)},
	inputencoding=utf8
}

% Eine weitere Stufe für Unterkapitel

% Definiere das Gliederungselement Subsubsubsection
\makeatletter
\newcommand\subsubsubsection{\@startsection{subsubsubsection}{4}{\z@}%
	{-3.25ex \@plus -1ex \@minus -.2ex}%
	{1.5ex \@plus .2ex}%
	{\normalfont\normalsize\bfseries}}
\newcounter{subsubsubsection}[subsubsection]
\renewcommand\thesubsubsubsection{\thesubsubsection.\@arabic\c@subsubsubsection}
\newcommand*\l@subsubsubsection{\@dottedtocline{3}{10.0em}{4.1em}}
\newcommand*{\subsubsubsectionmark}[1]{}
\makeatother

% Füge das Gliederungselement dem Inhaltsverzeichnis hinzu
\setcounter{secnumdepth}{4}
\setcounter{tocdepth}{4}

% Dieser von Stackoverflow stammende Block sorgt dafür, dass Umlaute in Code-Listings funktionieren.
% Quelle:  https://tex.stackexchange.com/questions/24528/having-problems-with-listings-and-utf-8-can-it-be-fixed
\lstset{
	inputencoding = utf8,  % Input encoding
	extendedchars = true,  % Extended ASCII
	literate      =        % Support additional characters
	{á}{{\'a}}1  {é}{{\'e}}1  {í}{{\'i}}1 {ó}{{\'o}}1  {ú}{{\'u}}1
	{Á}{{\'A}}1  {É}{{\'E}}1  {Í}{{\'I}}1 {Ó}{{\'O}}1  {Ú}{{\'U}}1
	{à}{{\`a}}1  {è}{{\`e}}1  {ì}{{\`i}}1 {ò}{{\`o}}1  {ù}{{\`u}}1
	{À}{{\`A}}1  {È}{{\`E}}1  {Ì}{{\`I}}1 {Ò}{{\`O}}1  {Ù}{{\`U}}1
	{ä}{{\"a}}1  {ë}{{\"e}}1  {ï}{{\"i}}1 {ö}{{\"o}}1  {ü}{{\"u}}1
	{Ä}{{\"A}}1  {Ë}{{\"E}}1  {Ï}{{\"I}}1 {Ö}{{\"O}}1  {Ü}{{\"U}}1
	{â}{{\^a}}1  {ê}{{\^e}}1  {î}{{\^i}}1 {ô}{{\^o}}1  {û}{{\^u}}1
	{Â}{{\^A}}1  {Ê}{{\^E}}1  {Î}{{\^I}}1 {Ô}{{\^O}}1  {Û}{{\^U}}1
	{œ}{{\oe}}1  {Œ}{{\OE}}1  {æ}{{\ae}}1 {Æ}{{\AE}}1  {ß}{{\ss}}1
	{ẞ}{{\SS}}1  {ç}{{\c{c}}}1 {Ç}{{\c{C}}}1 {ø}{{\o}}1  {Ø}{{\O}}1
	{å}{{\aa}}1  {Å}{{\AA}}1  {ã}{{\~a}}1  {õ}{{\~o}}1 {Ã}{{\~A}}1
	{Õ}{{\~O}}1  {ñ}{{\~n}}1  {Ñ}{{\~N}}1  {¿}{{?`}}1  {¡}{{!`}}1
	{°}{{\textdegree}}1 {º}{{\textordmasculine}}1 {ª}{{\textordfeminine}}1
	{£}{{\pounds}}1  {©}{{\copyright}}1  {®}{{\textregistered}}1
	{«}{{\guillemotleft}}1  {»}{{\guillemotright}}1  {Ð}{{\DH}}1  {ð}{{\dh}}1
	{Ý}{{\'Y}}1    {ý}{{\'y}}1    {Þ}{{\TH}}1    {þ}{{\th}}1    {Ă}{{\u{A}}}1
	{ă}{{\u{a}}}1  {Ą}{{\k{A}}}1  {ą}{{\k{a}}}1  {Ć}{{\'C}}1    {ć}{{\'c}}1
	{Č}{{\v{C}}}1  {č}{{\v{c}}}1  {Ď}{{\v{D}}}1  {ď}{{\v{d}}}1  {Đ}{{\DJ}}1
	{đ}{{\dj}}1    {Ė}{{\.{E}}}1  {ė}{{\.{e}}}1  {Ę}{{\k{E}}}1  {ę}{{\k{e}}}1
	{Ě}{{\v{E}}}1  {ě}{{\v{e}}}1  {Ğ}{{\u{G}}}1  {ğ}{{\u{g}}}1  {Ĩ}{{\~I}}1
	{ĩ}{{\~\i}}1   {Į}{{\k{I}}}1  {į}{{\k{i}}}1  {İ}{{\.{I}}}1  {ı}{{\i}}1
	{Ĺ}{{\'L}}1    {ĺ}{{\'l}}1    {Ľ}{{\v{L}}}1  {ľ}{{\v{l}}}1  {Ł}{{\L{}}}1
	{ł}{{\l{}}}1   {Ń}{{\'N}}1    {ń}{{\'n}}1    {Ň}{{\v{N}}}1  {ň}{{\v{n}}}1
	{Ő}{{\H{O}}}1  {ő}{{\H{o}}}1  {Ŕ}{{\'{R}}}1  {ŕ}{{\'{r}}}1  {Ř}{{\v{R}}}1
	{ř}{{\v{r}}}1  {Ś}{{\'S}}1    {ś}{{\'s}}1    {Ş}{{\c{S}}}1  {ş}{{\c{s}}}1
	{Š}{{\v{S}}}1  {š}{{\v{s}}}1  {Ť}{{\v{T}}}1  {ť}{{\v{t}}}1  {Ũ}{{\~U}}1
	{ũ}{{\~u}}1    {Ū}{{\={U}}}1  {ū}{{\={u}}}1  {Ů}{{\r{U}}}1  {ů}{{\r{u}}}1
	{Ű}{{\H{U}}}1  {ű}{{\H{u}}}1  {Ų}{{\k{U}}}1  {ų}{{\k{u}}}1  {Ź}{{\'Z}}1
	{ź}{{\'z}}1    {Ż}{{\.Z}}1    {ż}{{\.z}}1    {Ž}{{\v{Z}}}1
	% ¿ and ¡ are not correctly displayed if inconsolata font is used
	% together with the lstlisting environment. Consider typing code in
	% external files and using \lstinputlisting to display them instead.      
}

% Definiere einige Stile für TikZ - Knoten
\tikzset{
	long node/.style={draw, rounded corners, minimum width=7.5cm, minimum height=1cm},
	short node/.style={draw, rounded corners, minimum width=2.5cm, minimum height=1cm},
	square node/.style={draw, rounded corners, minimum width=2.5cm, minimum height=2.5cm},
	label node/.style={font=\small, align=center}
}

% Abkürzungen ("Makros") können mit \newcommand definiert werden
% Symbole für die gängigen Mengen
\newcommand{\N}{\mathbb{N}}
\newcommand{\Z}{\mathbb{Z}}
\newcommand{\Zp}{\Z_{\geq 0}}
\newcommand{\Q}{\mathbb{Q}}
\newcommand{\Qp}{\Q_{\geq 0}}
\newcommand{\R}{\mathbb{R}}
\newcommand{\Rp}{\R_{\geq 0}}
\newcommand{\Rmn}{\R^{m\times n}}

% Hübscheres Epsilon
\newcommand{\eps}{\varepsilon}

% Einführung von TikZ - Bibliotheken
\usetikzlibrary{positioning,fit,shapes.geometric}

%%%%%%%%%%%%%%%%%%%%%%%%%%%%%%%%%%%%%%%%%%%%%%%%%%%
% Metadaten: Autor, Titel, etc.
\author{Yavuzâlp Dal}
\title{Einf\"uhrung in die Kerndichtesch\"atzung}
\subject{Schriftliche Ausarbeitung zum Seminarvortrag}
\def \vorlesung {Seminar: Angewandte Analysis}
\def \semester {Wintersemester 2025/26}
\def \dozent {Prof. Dr. Holger Schwender}
\date{Stand: \today}
%%%%%%%%%%%%%%%%%%%%%%%%%%%%%%%%%%%%%%%%%%%%%%%%%%%
